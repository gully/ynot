%\documentclass[twocolumn]{emulateapj}% change onecolumn to iop for fancy, iop to onecolumn for manuscript
\documentclass[twocolumn]{aastex63}
\bibliographystyle{aasjournal}

\usepackage{graphicx}
\usepackage[caption=false]{subfig}
\usepackage{amsmath}
\usepackage{lipsum}
\usepackage[]{mdframed}
\usepackage{booktabs}

\usepackage{fontspec}
\usepackage[T1]{fontenc}
\usepackage{newtxsf}
%\setmainfont{Fira Sans Book}[Scale=1.05]


\let\pwiflocal=\iffalse \let\pwifjournal=\iffalse



\shorttitle{Autodiff for 2D echellograms}
\shortauthors{Gully-Santiago}
%\bibliographystyle{yahapj}

\begin{document}
\title{Towards a Spectrograph Digital Twin: Forward Modeling Pixels of 2D Echellograms with Interpretable Machine Learning}

\author{Michael Gully-Santiago}
\affiliation{University of Texas at Austin Department of Astronomy}

\author{TBD}
\affiliation{TBD}


\begin{abstract}

  We introduce autodiff for 2D echellogram forward modeling.

\end{abstract}

\keywords{data: data analysis ---  stars: statistics}


\section{Introduction}\label{sec:intro}

Virtually all modern insights derived from observational optical spectroscopy first arrive through the lossy, imperfect, and unavoidable process of digitization onto a 2-dimensional (2D) focal plane array detector.  Moore's law has propelled the amount of spectrosopic information we can cram onto these detectors over the last 50 years, with the number of pixels catapulting up to a billion-fold.  Innovative optical designs have made evermore effective use of these pixels with information-rich multi-object, imager slicer, and \`echellogram spectral formats.  The computerized distillation of these spectrograms is generally turnkey.  Facility reduction pipelines provide the humdrum translation of raw 2D pixels to familiar 1D extracted spectra, ideally an afterhought in a practioner's journey from observation to scientific results.

But increasingly our greatest scientific ambitions are pushing against the limits of how much information we can squeeze out of a given spectrographic observation.  The discovery of other Earths, the redshift of the epoch of reionization, and some of the most important unrealized astronomical discoveries in a generation will occur in the margins of what our most powerful spectrographs can deliver.

The measurement of Extreme Precision Radial Velocity (EPRV) stands out as particularly susceptible to the vagueries of the spectral extraction process.  The instrumental Radial Velocity (RV) precision needed to detect an Earth-like planet around a Sun-like star amounts to a mere 1 cm/s, equivalent to a distance of 10 silicon atoms in a typical spectrograph pixel (cite HPF).  Sub-pixel flat fields (cite XX), charge transfer inefficiencies (cite XX), and other seemingly pathological materials science phenomena matter at these precisions.  Spectral extraction is \emph{not} turnkey at the level of 10 Si atoms.  These precision demands have catalyzed a renewed interest in understanding the delicate interplay of the incident stellar spectrum with the detector pixels.

Most 2D \'echellogram data reduction pipelines can be described as data-driven extraction procedures as opposed to interpretable scene models. The most auspicious of these procedures may be ``Optimal Extraction'' \citep{1986PASP...98..609H}, which treats each column as an independent example of a 1D line profile, enabling a judicious weighting in the cross-dispersion direction.  This prescription faithfully incorporates low-but-significant Signal-to-Noise Ratio (SNR) pixels.  Translating Optimal Extraction to \'echelle spectra gets complicated by the dense and sometime overlapping layouts of the so-called \'echellogram format.  Spectral traces generally do not align with the pixel coordinates, often curving into smile-shaped arcs.  Specialized methods have been developed to handle such layouts \citep{2002A&A...385.1095P,2014A&A...561A..59Z,2020arXiv200805827P}.

The desire to reach the elusive photon-noise limit motivated a subtle mental leap compared to Optimal Extraction.  \emph{Spectroperfectionism} \citep{2010PASP..122..248B} represents a \emph{scene model}, allowing 2D PSF shapes with possibly asymmetric morphologies, convolved with an underlying super-resolution 1D spectrum with unknown spectral structure.  This 2D PSF better resembles the aberrations of a genuine optical system, albeit with the downsides of high computational cost and the inevitability of introducing artificial ringing artifacts into the deconvolved spectrum\footnote{\url{https://hoggresearch.blogspot.com/2019/10/complexifying-optimal-extraction.html}}.  The spectroperfectionism approach has recently been demonstrated on \'echelle spectra for PRV purposes, achieving comparable performance as optimal extraction for the MINERVA spectrograph \citep{2019PASP..131l4503C}.

These existing techniques generally suit the vast majority of use cases, but necessarily cannot handle all science applications.  For example, spatially resolved binaries with separations less than the spectrograph slit length can be acquired with alignment along a spectrograph entrance slit, nominally offering two spectra for the price of one.  But data acquired in this way would break most normal pipelines.  More generally, any extended source, such as a photo-dissociation regions (PDR), Solar System surfaces, or circumstellar Keplerian gas disks, possess an admixture of spatial and spectral information convolved together in a way that can make separation and interpretation difficult.  A hypothetical general-purpose spectroscopic scene smodel would be enormously helpful in these contexts, but few such solutions exist.  Spectroastrometry (cite XX) and the recent slitless scene model for CTIO \citep{2023arXiv230704898N} stand out as recent successes.

Brown dwarfs represent a particularly adversarial example for spectral extraction for two reasons.  Their intrinsically low luminosities mean that all but the closest brown dwarfs have low signal-to-noise ratio at high spectral resolution with existing facilities.  Their ultracool atmospheres contain many molecular absorption bands, in which large swaths of spectra plunge to near zero detectable flux.  The combination of low signal-to-noise ratio and large chunks of missing continuum means that the mere act of identifying a spectral trace becomes difficult, breaking core assumptions in Optimal Extraction.  An even more extreme example---emission line sources---lack a continuum entirely, defying the common pipeline assumption of some detectable spectrum in each pixel sequence.  Heuristics for handling these special cases may exist, but they are often bespoke afterthoughts or require user intervention to guide the process along.

Even in EPRV, the assumptions in optimal extraction are not actually optimal, and the choices in spectroperfectionism not actually perfect.  Optimal extraction generally treats the cross-dispersion direction as exactly aligned with the detector $y$ axis, a computationally expedient heuristic. The RV bias of optimal extraction grows as the spectral traces tilt and bend farther from the detector pixel alignment.  Straight-but-tilted traces have the effect of polluting adjacent wavelength bins in nearly equal measure and, to first order, lowering the effective spectral resolving power.  Curved \'echelle traces are more insidious.  Here, the smearing of pixels can be expected to proceed with a systematic RV shift, since the shoulder of one spectral line notices more of its pixel neighbor to the left or right, depending on the extent and direction of the spectral trace's curvature.  These seemingly pathological minor biases must make a difference at the 10 Si atom level, but quantifying such complex effects has been elusive in part due to the lack of an end-to-end simulation framework for \'echellograms.

In this work we put forth a framework for a spectrograph \'echellogram \emph{digital twin}.  This term refers to a class of extreme forward modeling--a digital twin should represent all or most of the relevant behaviors of a complex system in a way that can be interrogated to build novel insights.  A digital twin should resemble---as close as possible---the actual state of the system at any given moment, providing simulated data that is indistinguishable from a genuine observation.  An \'echelle spectrograph digital twin therefore builds upon and transcends the notions of scene modeling introduced in \citet{2010PASP..122..248B}, by aspiring to include evermore realistic attributes of the spectrograph's imaging system into the computational machinery.  A digital twin would improve upon these limitations by providing a more physically realistic generative model for the underlying spectrum.

The challenge of creating a spectrograph digital twin are enormous.  Generally only a few individual practitioners possesses the requisite knowledge of all the copious ways a spectrograph may be perturbed, and an even smaller fraction of such individuals have the capacity or desire to translate those perturbations into an evaluable forward scene model.  Even if such a model could be written down and codified into a program, evaluating that program is destined to be computational expensive owing to a litany of inner loops of sub-pixel simulations.  Any likelihood function would need to be re\:evaluated as a Bayesian inverse problem to determine the settings of the tunable scene model.  The spectrograph state, sky emission lines, and underlying target star spectrum may have tens of thousands of variables needed to adequately describe the spectrogram.  Tuning tens of thousands of non-linear parameters in this complex forward model would appear hopelessly intractable.  Even if one could somehow linearize some large subset of these parameters via linear algebra \citet{2010PASP..122..248B}, correlated residual structures would inevitably predominate the likelihood calculation, hampering the application of computationally expedient chi-squared algorithms.

Machine learning innovations make digital twins newly possible. First, and most importantly, machine learning frameworks such as PyTorch (cite XX), Tensorflow (cite XX), JAX (cite XX), and others offer end-to-end atomatic differentiation.  This key enabling technology makes it possible to tune an unlimited number of non-linear parameters through the process of ``backwards propagation'', or simply \emph{backprop}.  Second, these frameworks all offer hardware acceleration, including Graphical Processing Units (GPUs) and Tensor Processing Units (TPUs).  These accelerators can yield 60-100$\times$ speedups with no code change.  Third, neural network architectures and/or kernel methods offer a strategy for learning patterns in correlated residual structures, which has the effect of making likelihood functions tolerant to inevitable imperfections in the scene model.  All together these and other modern computational innovations unlock a truly new category of performance



\textbf{The challenge: instrumental defects and imperfections.}


\begin{mdframed}
  \textbf{Common assumptions embedded into current techniques} \par
  - Single point source\par
  - Source has continuum\par
  - Continuum is high SNR\par
  - Relatively few segments of uninterrupted continuum\par
  - Trends adequately captured by polynomials\par
  - Symmetric, Gaussian-like PSF\par
\end{mdframed}

\begin{mdframed}
  \textbf{Extreme or unusual spectra break these assumptions} \par
  - brown dwarfs with low SNR and highly structured/missing spectra\par
  - emission line spectra (no continuum)\par
  - binary stars on the same slit\par
  - extended objects (non-point sources)\par
  - EPRV\par
\end{mdframed}

\begin{mdframed}
  \textbf{Many heuristics designed to cope with departures from these assumptions} \par
  \textcolor{lightgray}{\lipsum[4]}
\end{mdframed}

\begin{mdframed}
  \textbf{Philosphy: Each previous spectrum should inform future spectra} \par
  Treat the instrument as a breathing, dynamic system\par
  \textcolor{lightgray}{\lipsum[5]}
\end{mdframed}

\begin{mdframed}
  \textbf{The epic hero: autodiff and GPUs, PyTorch, flexible models} \par
  - Why this is only recently possible\par
  - Why this may be in some ways easier to reason about than heuristics\par
  - This work: A new autodiff-aware 2D echellogram modeling framework\par
  \textcolor{lightgray}{\lipsum[6]}
\end{mdframed}


\section{Methodology: Modeling 2D Pixels}

\begin{mdframed}
  \textbf{A mapping of $(x, y)$ pixel coordinates to $(\lambda, s)$ physical coordinates} \par
  \textcolor{lightgray}{\lipsum[7]}
\end{mdframed}

\begin{mdframed}
  \textbf{How to represent the target spectrum} \par
  - How to represent the sky spectrum\par
  - How to represent the target PSF\par
  - How to represent the sky spatial extent\par
  - How to represent the slit\par
\end{mdframed}


\subsection{Constructing a Resilient Likelihood}
\begin{mdframed}
  \textbf{The need to joint model, regularize, address outliers} \par
  - The likelihood function and per-pixel uncertainties\par
  - Arcs: sparsely encode both wavelength and slit position\par
  - Flats: encode just slit position\par
  - Darks: encode just background\par
  - Target spectra: encode wavelength, slit position, target position\par
  \textcolor{lightgray}{\lipsum[7]}
\end{mdframed}


\section{Results 1: Training on Synthetic Data with Injection/Recovery Tests}

\begin{mdframed}
  \textbf{Injection/recovery test with noisy data: generating fake data} \par
  \textcolor{lightgray}{\lipsum[9]}
\end{mdframed}

\begin{mdframed}
  \textbf{Initializing of the model and optimization setup} \par
  \textcolor{lightgray}{\lipsum[10]}
\end{mdframed}


\begin{mdframed}
  \textbf{Training computational performance} \par
  - Number of epochs\par
  - Batching/sparsity\par
  \textcolor{lightgray}{\lipsum[9]}
\end{mdframed}

\begin{mdframed}
  \textbf{Best fit model comparison 1: injection/recovery of initial parameters} \par
  - Number of epochs\par
  - Batching/sparsity\par
  \textcolor{lightgray}{\lipsum[10]}
\end{mdframed}

\begin{mdframed}
  \textbf{Best fit model comparison 2: spectrum as unbinned, weighted samples} \par
  - Number of epochs\par
  - Batching/sparsity\par
  \textcolor{lightgray}{\lipsum[11]}
\end{mdframed}


\section{Results 2: Training on real data}
\begin{mdframed}
  \textbf{Introduction to real data} \par
  - Data-preprocessing and heuristics \par
  \textcolor{lightgray}{\lipsum[12]}
\end{mdframed}

\begin{mdframed}
  \textbf{Outcome: reduced spectrum as unbinned, weighted samples} \par
  - SNR improvement compared to previous methods (head-to-head)\par
  \textcolor{lightgray}{\lipsum[13]}
\end{mdframed}

\pagebreak
\clearpage

\section{Discussion}
\begin{mdframed}
  \textbf{The promise for EPRV} \par
  - Simulation of minor RV shifts\par
  - Simulation of sub-pixel flat fields \par
  - Tracking spectrograph state across decades of operation\par
  \textcolor{lightgray}{\lipsum[14]}
\end{mdframed}


\begin{mdframed}
  \textbf{Ability to repurpose non-standard data (variable data quality)} \par
  \textcolor{lightgray}{\lipsum[15]}
\end{mdframed}


\begin{mdframed}
  \textbf{Conceivable extensions} \par
  \textcolor{lightgray}{\lipsum[16]}
\end{mdframed}



\acknowledgements


\facilities{Keck (NIRSPEC), Gaia}

\software{  pandas \citep{mckinney10, reback2020pandas},
  matplotlib \citep{hunter07},
  numpy \citep{harris2020array},
  scipy \citep{jones01},
  ipython \citep{perez07},
  pytorch \citep{NEURIPS2019_9015}}

\appendix

\section{Comparison to other frameworks}

Hundreds or possibly thousands of 2D extraction pipelines have been written and rewritten to suit the disparate needs of each unique spectrograph.  Table 1 of \citet{2022PASP..134k4509C} lists 18 \'echelle spectrographs and their data reduction pipelines.  Many but not all of those pipelines are open source.  Here we list a subset of recent open source pipelines for \'echelle spectra reduction.

Few approaches have attempted to combine spectral scene modeling in combination with machine learning techniques.  \citet{2020MNRAS.499.1972X} employed a 2D PSF convolution with neural networks on a demonstration with multi-object spectra from LAMOST.  While promising, the code is not open source and the applications limited to inferring the field distortion, but not extending to subsequent extraction steps.

\citet{2019PASP..131l4503C} developed spectroperfectionism for the MINERVA spectrograph, finding similar performance as optimal extraction.


\begin{deluxetable*}{lllll}
  \tablewidth{0pc}
  \tablecaption{
    Selection of open source codes for 2D echelle data reduction
    \label{tabHeSM}
  }
  \tablehead{
    \colhead{Code}   &
    \colhead{Instrument} &
    \colhead{Reference} &
    \colhead{Language} &
    \colhead{Note}
  }
  \startdata
  \hline
  \multicolumn{5}{c}{\'echelle} \\
  \hline
  \href{https://github.com/igrins/plp}{PLP} & IGRINS & \citet{2014AdSpR..53.1647S, jaejoonlee15} & Python & \\
  \href{https://github.com/AWehrhahn/PyReduce}{PyReduce} & HARPS, UVES & \citet{2021AA...646A..32P} & Python & \\
  \href{https://github.com/njcuk9999/apero-drs}{APERO} & SPIRou & \citet{2022PASP..134k4509C} & Python &\\
  \hline
  \multicolumn{5}{c}{Long-slit} \\
  \hline
  \href{https://github.com/LSSTDESC/Spectractor}{Spectractor} & CTIO & \citet{2023arXiv230704898N} & Python & Scene Modeling\\
  \enddata
  \tablecomments{Some codes are applicable to multiple instruments, and some instruments have multiple codes.}
\end{deluxetable*}




\clearpage

\bibliographystyle{apj}
\bibliography{ms}

\end{document}
