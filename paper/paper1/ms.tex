%\documentclass[twocolumn]{emulateapj}% change onecolumn to iop for fancy, iop to onecolumn for manuscript
\documentclass[twocolumn]{aastex63}
\bibliographystyle{aasjournal}

\usepackage{graphicx}
\usepackage[caption=false]{subfig}
\usepackage{amsmath}
\usepackage{lipsum}
\usepackage[]{mdframed}
\usepackage{booktabs}

\usepackage{fontspec}
\usepackage[T1]{fontenc}
\usepackage{newtxsf}
%\setmainfont{Fira Sans Book}[Scale=1.05]


\let\pwiflocal=\iffalse \let\pwifjournal=\iffalse



\shorttitle{Autodiff for 2D echellograms}
\shortauthors{Gully-Santiago}
%\bibliographystyle{yahapj}

\begin{document}
\title{Towards a Spectrograph Digital Twin: Forward Modeling Pixels of 2D Echellograms with Interpretable Machine Learning}

\author{Michael Gully-Santiago}
\affiliation{University of Texas at Austin Department of Astronomy}

\author{TBD}
\affiliation{TBD}


\begin{abstract}

  We introduce autodiff for 2D echellogram forward modeling.

\end{abstract}

\keywords{data: data analysis ---  stars: statistics}


\section{Introduction}\label{sec:intro}

Virtually all modern insights derived from observational optical spectroscopy first arrive through the lossy, imperfect, and unavoidable process of digitization onto a 2-dimensional focal plane array detector.  Moore's law has propelled the amount of spectrosopic information we can cram onto these detectors over the last 50 years, with the number of pixels catapulting up to a billion-fold.  Innovative optical designs have made evermore effective use of these pixels with information-rich multi-object, imager slicer, and \`echellogram spectral formats.  The computerized distillation of the target spectrum is generally turnkey.  Facility reduction pipelines provide the humdrum translation of raw 2D pixels to familiar 1D extracted spectra, ideally an afterhought in a practioner's journey from observation to scientific results.

But increasingly our greatest scientific ambitions are pushing against the limits of how much information we can squeeze out of a given spectrographic observation.  The discovery of other Earths, the redshift of the epoch of reionization, and some of the most important unrealized astronomical discoveries in a generation will occur in the margins of what our most powerful spectrographs---even those on \emph{JWST}---can deliver.

The measurement of Extreme Precision Radial Velocity (EPRV) stands out as particularly susceptible to the vagueries of the spectral extraction process.  The instrumental Radial Velocity (RV) precision needed to detect an Earth-like planet around a Sun-like star amounts to a mere 1 cm/s, equivalent to a distance of 10 silicon atoms in a typical spectrograph pixel (cite HPF).  Spectral extraction is \emph{not} turnkey at the level of 10 Si atoms.  These precision demands have catalyzed a renewed interest in understanding the delicate interplay of the incident stellar spectrum with the detector pixels.  We seek a holistic view of 2D \'echellogram modeling.

Most 2D \'echellogram data reduction pipelines can be described as procedural: the sequential application of operations with the output of the previous steps serving as intermediate checkpoints and inputs to the next steps.  ``Optimal Extraction'' \citep{1986PASP...98..609H} introduced a separable (1D) Gaussian weighting in the cross-dispersion direction, faithfully incorporating low-but-significant Signal-to-Noise Ratio (SNR) pixels.  \'Echellogram spectral traces generally do not align with the pixel coordinates, often bowing into smile-shaped curves and complicating optimal extraction \citep{2002A&A...385.1095P,2020arXiv200805827P}.

The desire to reach the photon-noise limit motivated a class of algorithms referred to as ``spectro-perfectionism'' \citep{2010PASP..122..248B}, which introduced a subtle mental leap compared to Optimal Extraction.  Rather than treat the PSF as adjacent, indepedent, semi-separable 1D Gaussians in the cross-dispersion direction, \citet{2010PASP..122..248B} allowed arbitrary 2D PSF shapes with unknown morphologies. This arbitrary 2D PSF better resembles the aberrations of a genuine optical system.  Flat-relative optimal extraction \citep{2014A&A...561A..59Z} avoided the need to model the actual PSF by introducing a data-driven scheme for weighting the pixels.

These existing techniques generally suit the vast majority of scientific applications, but not all.  The techniques make assumptions---either implicitly or explicitly---that begin to degrade for unusual spectra...

Even for EPRV, the assumptions in optimal extraction are not optimal.  These approaches generally treat the cross-dispersion direction as exactly aligned with the detector $y$ axis, which has the effect of polluting adjacent wavelength bins and lowering the resolution slightly.  For example, the \citet{2010PASP..122..248B} modeling technique can be thought of as a deconvolution process, and therefore suffers from ringing artifacts in the inferred super-resolution spectrum\footnote{\url{https://hoggresearch.blogspot.com/2019/10/complexifying-optimal-extraction.html}}.


\textbf{The challenge: instrumental defects and imperfections.}

The essence of 2D spectral extraction is familiar to most practitioners: a monochromatic unresolved point source looks like a free-standing Point Spread Function (PSF), possibly truncated by an entrance slit into a rounded rectangle.  A polychromatic point source convolves this PSF with some line segment

\begin{mdframed}
  \textbf{Common assumptions embedded into current techniques} \par
  - Single point source\par
  - Source has continuum\par
  - Continuum is high SNR\par
  - Relatively few segments of uninterrupted continuum\par
  - Trends adequately captured by polynomials\par
  - Symmetric, Gaussian-like PSF\par
\end{mdframed}

\begin{mdframed}
  \textbf{Extreme or unusual spectra break these assumptions} \par
  - brown dwarfs with low SNR and highly structured/missing spectra\par
  - emission line spectra (no continuum)\par
  - binary stars on the same slit\par
  - extended objects (non-point sources)\par
  - EPRV\par
\end{mdframed}

\begin{mdframed}
  \textbf{Many heuristics designed to cope with departures from these assumptions} \par
  \textcolor{lightgray}{\lipsum[4]}
\end{mdframed}

\begin{mdframed}
  \textbf{Philosphy: Each previous spectrum should inform future spectra} \par
  Treat the instrument as a breathing, dynamic system\par
  \textcolor{lightgray}{\lipsum[5]}
\end{mdframed}

\begin{mdframed}
  \textbf{The epic hero: autodiff and GPUs, PyTorch, flexible models} \par
  - Why this is only recently possible\par
  - Why this may be in some ways easier to reason about than heuristics\par
  - This work: A new autodiff-aware 2D echellogram modeling framework\par
  \textcolor{lightgray}{\lipsum[6]}
\end{mdframed}


\section{Methodology: Modeling 2D Pixels}

\begin{mdframed}
  \textbf{A mapping of $(x, y)$ pixel coordinates to $(\lambda, s)$ physical coordinates} \par
  \textcolor{lightgray}{\lipsum[7]}
\end{mdframed}

\begin{mdframed}
  \textbf{How to represent the target spectrum} \par
  - How to represent the sky spectrum\par
  - How to represent the target PSF\par
  - How to represent the sky spatial extent\par
  - How to represent the slit\par
\end{mdframed}


\subsection{Constructing a Resilient Likelihood}
\begin{mdframed}
  \textbf{The need to joint model, regularize, address outliers} \par
  - The likelihood function and per-pixel uncertainties\par
  - Arcs: sparsely encode both wavelength and slit position\par
  - Flats: encode just slit position\par
  - Darks: encode just background\par
  - Target spectra: encode wavelength, slit position, target position\par
  \textcolor{lightgray}{\lipsum[7]}
\end{mdframed}


\section{Results 1: Training on Synthetic Data with Injection/Recovery Tests}

\begin{mdframed}
  \textbf{Injection/recovery test with noisy data: generating fake data} \par
  \textcolor{lightgray}{\lipsum[9]}
\end{mdframed}

\begin{mdframed}
  \textbf{Initializing of the model and optimization setup} \par
  \textcolor{lightgray}{\lipsum[10]}
\end{mdframed}


\begin{mdframed}
  \textbf{Training computational performance} \par
  - Number of epochs\par
  - Batching/sparsity\par
  \textcolor{lightgray}{\lipsum[9]}
\end{mdframed}

\begin{mdframed}
  \textbf{Best fit model comparison 1: injection/recovery of initial parameters} \par
  - Number of epochs\par
  - Batching/sparsity\par
  \textcolor{lightgray}{\lipsum[10]}
\end{mdframed}

\begin{mdframed}
  \textbf{Best fit model comparison 2: spectrum as unbinned, weighted samples} \par
  - Number of epochs\par
  - Batching/sparsity\par
  \textcolor{lightgray}{\lipsum[11]}
\end{mdframed}


\section{Results 2: Training on real data}
\begin{mdframed}
  \textbf{Introduction to real data} \par
  - Data-preprocessing and heuristics \par
  \textcolor{lightgray}{\lipsum[12]}
\end{mdframed}

\begin{mdframed}
  \textbf{Outcome: reduced spectrum as unbinned, weighted samples} \par
  - SNR improvement compared to previous methods (head-to-head)\par
  \textcolor{lightgray}{\lipsum[13]}
\end{mdframed}

\pagebreak
\clearpage

\section{Discussion}
\begin{mdframed}
  \textbf{The promise for EPRV} \par
  - Simulation of minor RV shifts\par
  - Simulation of sub-pixel flat fields \par
  - Tracking spectrograph state across decades of operation\par
  \textcolor{lightgray}{\lipsum[14]}
\end{mdframed}


\begin{mdframed}
  \textbf{Ability to repurpose non-standard data (variable data quality)} \par
  \textcolor{lightgray}{\lipsum[15]}
\end{mdframed}


\begin{mdframed}
  \textbf{Conceivable extensions} \par
  \textcolor{lightgray}{\lipsum[16]}
\end{mdframed}



\acknowledgements


\facilities{Keck (NIRSPEC), Gaia}

\software{  pandas \citep{mckinney10, reback2020pandas},
  matplotlib \citep{hunter07},
  numpy \citep{harris2020array},
  scipy \citep{jones01},
  ipython \citep{perez07},
  pytorch \citep{NEURIPS2019_9015}}



\clearpage

\bibliographystyle{apj}
\bibliography{ms}

\end{document}
